\chapter{Summary \& Conclusion}

\section{Continuing Work}
Deployment in field, over the air firmware update, http server on onway hardware, cost reduction of hardware, accelerometer
\todo[inline]{Finalize this text}


\section{Reflection \& Project Schedule}
\todo[inline]{Finalize this text}
\todo[inline]{Write about problems we had with ESP-IDF and thus changed to the Arduino framework}
milestones reached as scheduled, setting up server/Usb took longer than expected, gained a lot of knowledge building a complete product.  
We developed a complete system based on our given specifications. Additional software tools were created to support our hardware in its operation. As a whole, the \acrfull{fms} Monitor operates normally and is reliable. 
Our time management was excellent, all milestones were reached as scheduled and all functions were implemented on time. There were some parts of the firmware that turned out to be more complicated than expected, resulting in long workdays in order to meet the schedule. Particularly, assembly and testing of the hardware took almost three times longer than we originally planned. This was caused by a small oversight in the schematic of the device, which led to a bad connection. After having rectified the problem in the Ethernet controller circuit, everything else went smoothly.

\newpage
\section{Personal Reflections}

\subsubsection{Florian Baumgartner}
This student research project allowed me to get involved to the \acrshort{can} interface and the \acrshort{fms} protocol. Further I could make use of my previously gained knowledge in the field of embedded systems and software engineering. It was a very positive experience to develop a fully working product in such a small time frame. The detailed planing of the project proved to be very important. Thus, I'm especially proud of meeting each milestone on time and preventing major delays. \newline
It was a pleasure to work with Luca Jost and we had overall a great time working on this project. My personal highlight was learning the PyQt5 framework as well as using the Plotly library. I'm fairly interested in the field of \acrshort{iot} devices, therefore this project was a great opportunity to deepen my knowledge and getting more experienced.

I'm glad that the hardware development went that smoothly although the ongoing worldwide chip shortage made it difficult to get access to the components needed. It payed out to premature focus on this particularly challenging situation and chose parts that were easily available. All in all, good communication was key to lead to a successful result.

\subsubsection{Luca Jost}
In general, this student research project overall has been very enjoyable. Within just a few weeks, we transformed an idea into a deployment-ready product. The Fleet-Monitor is working as designed and I am looking forward to seeing the device being used in the field. During the project I was able to leverage my previous experience and build on it. I found it particularly fascinating to learn more about \acrshort{can} systems, as it is a very popular technology in the industry currently. Working with embedded systems is something I have always liked and this project was no exception. Developing real-time operating systems with complex logic is something I enjoy. This project was not particularly complex, and unfortunately there was nothing that challenged me significantly.

In past projects, I had trouble with time management, so we made sure to develop a realistic timetable this time. Spending the extra time on the schedule has proven to be very advantageous as I was able to deliver all tasks on time. 

The experience of working with Florian Baumgarnter was extremely rewarding; his dedication and attention to detail are admirable. 