\begin{abstract}
Onway AG offers WLAN and network access control solutions and software development. Their main fields of business are solutions for Network Access Control (NAC) as well as communication access for public transport. They are known for developing specialized industrial IoT applications. Onway AG is interested in providing an elegant solution for public transport fleets (e.g. buses) to gather low-level vehicle data and transmit them to a cloud-based system. This information can then be used to monitor the state of the vehicle and inform about possible issues in real time.

The Fleet Management Systems Interface (FMS) is a standard interface developed by European commercial vehicle manufacturers in 2002. It defines a common interface for telematics applications and include driving as well as diagnostics information. The data is coded onto a CAN-Bus. This semester project introduces an FMS monitoring system which transmits real time data from the vehicle to a server. \\ The Fleet-Monitor Hardware was designed from scratch with the goal to make it simple, reliable and deployable in a vehicle. The system is based around an ESP32-S2 system on chip and all the Software is written in C++. The device connects directly to the CAN-Bus, collects and filters incoming data and forwards it to a host device over Ethernet or WiFi. Additionally, an accelerometer was added to monitor information about the driving performance. A file system was implemented to easily access the configuration. The configuration can be uploaded via the USB port or over the air.

The hardware was then tested to assess its functionality. Recorded data from a bus driving in Germany, as well as a J1939 simulator were used to evaluate the system performance. These tests showed that the system is promising and fulfils all of the given requirements.
\end{abstract}


%% Temporary!
\subsubsection{General TODOs}
\todo[inline]{Luca, Add licensing block to all firmware code files}
\todo[inline]{Florian, Create image of SD-Card from raspy and backup data on GitHub}
\todo[inline]{Add "unbreakable" space between value and unit (e.g. 69.0\,V), syntax is: \textbackslash,}

%\clearpage