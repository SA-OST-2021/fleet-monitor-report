\chapter{Preliminaries}
%%% Local Variables:
%%% mode: latex
%%% TeX-master: "thesis"
%%% End:




\section{Controller Area Network (CAN)}
ISO11898 describes the physical data link layer implementation of \acrshort{can}. This specification describes a twisted-wire pair bus with 120$\Omega$ line impedance, and differential signaling at a rate up to 1 Mbps. A network is constructed with two or more transceivers on the same bus lines. The network must be terminated with a 120$\Omega$ resistor at each end of the bus, as shown in Figure \ref{fig:can-bus_topology}.


\begin{figure}[h!]
	\centering
	\includegraphics[height=4cm]{images/can-bus_topology}
	\caption{Topology of a \acrshort{can}-Bus}
	\vspace{-1.4ex}
	\label{fig:can-bus_topology}
\end{figure}

As long as the bus is free, any node is allowed to transmit \acrshort{can} messages. Each message is received by all nodes on the network, including the node that sent the message. This type of data broadcasting allows multiple nodes to use the transmitted data. It also allows the sending node to monitor the bus for errors. If two or more nodes try to transmit at the same time, the lower priority message will be overwritten, and this lower priority node will halt transmission upon sensing overwritten bits in its message identifier. The message is then re-transmitted when the bus is again free. This non-destructive process is called bit-wise arbitration.

Every node on the network reads the identifier of a message, and each node independently determines if the message is to be ignored or processed. Since the identifier is specific to the contents of the message rather than the identity of the originating node, new nodes may be added to the network, without modifying the program of any existing node on the network.

Source: https://www.ti.com/lit/an/slla109a/slla109a.pdf,
https://www.kvaser.com/can-protocol-tutorial/

\section{SAE J1939}
J1939 is the open standard developed by \acrfull{sae}. It is used for networking and communication in the commercial vehicle sector. J1939 is a higher-layer protocol utilizing \acrshort{can} as its physical layer. SAE J1939 is primarily a data driven protocol, providing far better data bandwidth than other automation protocols such as CANopen and DeviceNet. 

The standard specifies CAN bus speeds of 250 kbit/s or 500 kbit/s and uses the extended 29-bit identifier. Most messages defined by the J1939 standard are intended to be broadcast. This means that the data is transmitted on the network without a specific destination. This permits any device to use the data without requiring additional request messages. The 29-bit identifier used in J1939 is structured in the following way.

\begin{figure}[h!]
	\centering
	\includegraphics[height=1.5cm]{images/j1939-identifier}
	\caption{29-bit J1939 Identifier}
	\vspace{-1.4ex}
	\label{fig:can-bus_topology}
\end{figure}

The first three bits of the identifier are used for controlling a message priority during the arbitration process. A value of 0 has the highest priority.

The next bit of the identifier is reserved for future use and is set to 0 for transmitted messages.

The next bit in the identifier is the data page selector. This bit expands the number of possible Parameter Groups that can be represented by the identifier.

The PDU format (PF) determines whether the message can be transmitted with a destination address or if the message is always transmitted as a broadcast message.

The interpretation of the PDU specific (PS) field changes based on the PF value:

If the PF is between 0 and 239, the message is addressable (PDU1) and the PS field contains the destination address.
If the PF is between 240 and 255, the message can only be broadcast (PDU2) and the PS field contains a Group Extension.
The Group extension expands the number of possible broadcast Parameter Groups that can be represented by the identifier.

The term Parameter Group Number (PGN) is used to refer to the value of the Reserve bit, DP, PF, and PS fields combined into a single 18 bit value.
\todo{this is just copied in for now, think about if we really need that}

Source: https://copperhilltech.com/a-brief-introduction-to-the-sae-j1939-protocol/,
https://www.kvaser.com/about-can/higher-layer-protocols/j1939-introduction/

\section{Fleet Monitoring System (FMS)}

